\documentclass[11pt]{amsart}
\usepackage{amsmath,amssymb,graphicx,color,setspace,enumerate,natbib}
\newcommand{\subscript}[2]{$#1#2$}
\newcommand{\Eyx}{\mathbb{E}_{Y|X}}
\newcommand{\Ex}{\mathbb{E}_X}
\newcommand{\sm}{\sup_{x \in M_c}}
\newcommand{\R}{\mathcal{R}}
\newcommand{\C}{C}
\newcommand{\X}{X}
\newcommand{\CX}{\C^m \circ \X}
\newcommand{\Rtwo}{\mathbb{R}^2}
\newcommand{\ltwo}{\mathbb{L}^2}
\newcommand{\uc}{\mathbb{S}}
\DeclareMathOperator*{\argmin}{argmin}
%\theoremstyle{plain}
\newtheorem{theorem}{Theorem}
\theoremstyle{definition}
\newtheorem{remark}{Remark}
%\newtheorem*{proof}{Proof}
\begin{document}
\section{Shapes of closed curves}
Our interest is in the completion of shapes of partially observed closed curves, and their classification. This first requires us to adopt a suitable representation for the shape of a fully observed curve. We adopt a parametric representation of a closed curve by representing as an absolutely continuous function\footnote{Upon identification of $\uc$ with $[0,1] \cong \mathbb{R}/2\pi\mathbb{Z}$, a curve $\C:[0,1] \to \mathbb{R}^2$ is absolutely continuous if and only if there exists an integrable function $g:[0,1]\to \Rtwo$ such that $\C(t)-\C(0)=\int_0^tg(u)du, \forall t \in [0,1]$.} $\C: \uc \to \Rtwo$, thus automatically ensuring that the curve is closed. The notion of a shape of such a curve requires invariances to transformations that represent nuisance information. Specifically, if $\Gamma:=\{\gamma:S \to S \text{ is an orientation-preserving diffeomorphism}\}$ and $SO(2)$ is the rotation group in $\mathbb{R}^3$, the shape of a parameterized curve $\C:D \to \Rtwo$ is defined to be the equivalence class 
\[
[\C]:=\Big\{\sigma O \C(\gamma(t))+a, \gamma \in \Gamma, O \in SO(2), a \in \Rtwo, \sigma >0\Big\}.
\]
Thus $[\C]$ is the set of all possible curves that can be obtained through a translation ($\C+a$), rotation ($O\C$), scale change $(\sigma \C)$, a reparametrization ($C(\gamma)$), or a combination of the transformations, of the curve $C$. In words, the shape of a curve $C$ is what is left once variations due scale, translation, rotation and reparametrisation has been accounted for. 


A key ingredient in several classification methods (e.g. linear/quadratic discriminant analysis; kernel-based methods) for functional data is the notion of similarity or distance between curves. A popular choice is the distance induced by the $\mathbb{L}^2$ norm of a Hilbert space of square-integrable functions. However, it is well known that the $\mathbb{L}^2$ is unsuitable for comparing curves in the presence of parameterisation variability \citep{KurtekJASA, AK}. To this end we employ a suitable representation (transformation) of a curve $\C$ that allows us to compute distances easily while accounting for the necessary invariances. 

\subsection{Notation}
A closed planar curve $C$ is always an absolutely continuous mapping $C:\uc \to \Rtwo$, and the set of planar closed curves will be denoted by $\mathcal{C}$. The set $\mathcal{C}$ is equipped with the norm $\|x\|_{\ltwo}:=[\int_\uc \|x(t)\|_2dt]^{1/2}$ where $\|\cdot\|_2$ is the Euclidean norm in $\Rtwo$. $SO(2)$ is the special orthogonal group of rotation matrices of $\Rtwo$, and $\Gamma$ is group of orientation-preserving diffeomorphisms of $\uc$. The set $\Gamma_I$ will denote the group $\{\gamma:[0,1]\to[0,1], \gamma'>0,\gamma(0)=0, \gamma(1)=1\}$. 
\subsection{Square-root velocity transform}
For a detailed introduction to the transform, its properties and advantages we refer to the reader to Chapter 6 of the book by \cite{AK}. Here we briefly outline the important concepts required for our purposes. 
For an absolutely continuous curve $\C:\uc \to \Rtwo$ consider the transformation 
$$\C \mapsto \frac{C'}{\|C'\|_{\ltwo}}=:Q_\C,$$ 
where $\C'$ is the (vector) derivative of $\C(t)$ with respect to $t$, and $\|\cdot\|$ is the usual Euclidean norm in $\Rtwo$. 

The unique (up to translations) inverse of a square-root transformed curve $Q_C$ is $\int_o^t Q_C(s)\|Q_C(s)\|_2ds$. To ensure that the curves are closed we need to impose an additional constraint: $\int_\uc \|C'(t)\|_2dt=\int_\uc Q_C(t)\|Q_C(t)\|_2=0$.  

For a curve $\C$, by taking its derivative and dividing by its length $\|C'\|_{\ltwo}$, the transform accounts for translation and scale variabilities. Thus the image of the set of absolutely continuous closed curves with fixed lengths $l$, $ \{\C:\uc \to \Rtwo: \int_\uc \|C'(t)\|_2dt=l\}$, under the square-root transform map $\C \mapsto Q_\C$ is the set
\[
\mathcal{Q}:=\Big\{Q_\C: \int_\uc \|Q_C\|_2^2=1, \int_\uc Q_C(t)\|Q_C(t)\|_2dt=0\Big\},
\]
since $\|Q_\C\|_{\ltwo}=\|C'\|_{\ltwo}^{1/2}$. Thus the set $\mathcal{Q}$ is a subset of $\mathbb{L}^2(\uc,\Rtwo):=\{Q:\uc \to \Rtwo: \int_\uc \|Q(t)\|_2^2 <\infty \}$. It is referred to as the pre-shape space corresponding to the curves, since variations due to rotation and parameterization are yet to be accounted for. It is not a linear space and is a manifold \citep{AK}. 

Before defining the shape space, we discuss the actions of groups $\Gamma$ and $SO(2)$ on the set $\mathcal{Q}$. The set $\Gamma$ of reparameterisations (or warp maps) of $\uc$ is group with group action given by composition. Its action on $\mathcal{Q}$ is defined by $(Q_\C,\gamma) \mapsto Q_\C(\gamma) \sqrt{|\gamma '|}$, where $\gamma'$ is the derivative of $\gamma$ (see Chapters 5 and 6 \cite{AK} for more details). The derivative of $\gamma:\uc \to \uc$ needs to be viewed as a derivative of $\gamma:[0,1]\to[0,1]$ based on the identification $\uc \cong \mathbb{R}/2\pi\mathbb{Z}$, and hence $|z|$ is just the absolute value of the real number $z$. 
The action of the rotation group $SO(2)$ is defined in the usual way as the map $SO(2) \times \mathcal{Q} \to \mathcal{Q}$ with $(O,Q_C)\mapsto \{OQ_C(t): t \in \uc\}$. 

Two important ramifications of the described framework, motivating its use in our work for analyzing shapes of curves, are the following. Under the square-root velocity framework:
\begin{enumerate}[1.]
\item The actions of $SO(2)$ and $\Gamma$ on $\mathcal{Q}$ commute, i.e. they can be applied to a curve in any order. This ensures that their combined action is given by the product group $\Gamma \times SO(2)$.
\item The action of $\Gamma \times SO(2)$ on $\mathcal{Q}$ is by isometries: Given two curves $\C_1$ and $\C_2$, we have $\|OQ_{\C_1}(\gamma)\sqrt{\gamma '}-OQ_{\C_2}(\gamma)\sqrt{\gamma '}\|_{\ltwo}=\|Q_{C_1}-Q_{C_2}\|_{\ltwo}$, for every $(\gamma,O) \in \Gamma \times SO(2)$. This ensures that if two square-root velocity transformed curves are rotated and reparameterized the same way, their distance remains unchanged. 
\end{enumerate}
 Starting with a curve $\C$ we can now define its shape to be the equivalence class or its orbit of its corresponding square-root transform:
 $$[Q_\C]=\text{closure}\{OQ_\C(\gamma) \sqrt{\gamma'}: (\gamma,O) \in \Gamma \times SO(2)\},$$
 where the closure is with respect to the norm $\|\cdot\|_{\ltwo}$ on $Q$. The shape space consequently is defined as $\mathcal{Q}_s:=\{[Q_\C]: Q_\C \in \mathcal{Q}\}$. Property (2) ensures that the metric induced by the norm $\|\cdot\|_{\ltwo}$ on $\mathcal{Q}$ descends onto a metric $d$ on the shape space (quotient space) $\mathcal{Q}_s$ in a natural way. Given two curves $\C_1$ and $\C_2$, the shape distance between them is defined as
 \begin{align}
 \label{distance}
 d(\C_1,\C_2)&:=\inf_{(\gamma,O) \in \Gamma \times SO(2)} \|Q_{\C_1} - OQ_{\C_2}(\gamma)\sqrt{\gamma'}\|_{\ltwo}\\
 &=\inf_{(\gamma,O) \in \Gamma \times SO(2)}\Big[\int_{\uc}\left\|Q_{C_1}(t)-OQ_{C_2}(\gamma(t))\sqrt{\gamma'(t)}\right\|_2^2dt\Big]^{1/2} \nonumber\\
 &=\inf_{(\gamma,O) \in \Gamma \times SO(2)} \|OQ_{C_1}(\gamma)\sqrt{\gamma'} - Q_{C_2}\|_{\ltwo}\nonumber.
  \end{align}
  The symmetry with respect to the action either on $Q_{C_1}$ or $Q_{C_2}$ is an attractive feature and will be used profitably in the sequel. 
 %%%%%%%%%%%%%%%%%%%%%%%%%%%%%%
 %%%%%%%%%%%%%%%%%%%%%%%%%%%%%%%
\section{Curve completion and classification}
Suppose we are given closed planar curves $\C^m_j:\uc \to \mathbb{R}^2,j=1\ldots,n$ each of which has been observed only on a region $\R_j \subset \uc$, where $\uc$ is the unit circle in $\Rtwo$. We assume that the curves are absolutely continuous. Additionally, a training sample $\{(y_i,\C^o_i),1,\ldots,N\}$ consisting of class labels $y_i \in \{0,1\}$ and $C^o_i:\uc\to \mathbb{R}^2$, fully observed on a common domain $\uc$, is provided. The set of fully observed curves are elements of the set $\mathcal{C}$; denote by $\mathcal{C}^m$ the set of all partially observed curves. 

The problem at hand is to model the shape of and complete each partially observed curve $\C^m_j$ and assign it to one of $G$ groups. We consider two approaches: one based on a variational formulation, and the other using kernel-based classifier. Before describing our approaches, some comments on the set $\Gamma$ are in order. 

\subsection{The set of reparameterizations $\Gamma$}
Elements of the group $\Gamma$ of diffeomorphisms of $\uc$ can be viewed in the following manner. The unit circle $\uc$ can be identified with the quotient group $\mathbb{R}/2\pi \mathbb{Z} \cong [0,1]$. Through this identification, every continuous mapping $\beta: \mathbb{R} \to \mathbb{R}$ induces a continuous mapping of $\uc$ onto itself such that $\beta(t+1)=\beta(t)+1$ for all $t \in \mathbb{R}$. If $\beta$ is monotone increasing, we say that the induced map on $\uc$ is orientation-preserving (based on a choice of clockwise or anti-clockwise orientation).

Consider now the set $$\Gamma_\mathbb{R}:=\{\beta:\mathbb{R}\to \mathbb{R}: \beta(t+1)=\beta(t)+1, \text{ continuous and increasing}\}.$$ Each member $\beta$ of $W_\mathbb{R}$ induces a warp map $\tilde{\beta}:\uc \to \uc$ with $\tilde{\beta}(e^{2\pi it})=e^{2\pi i\beta(t)}$, where $\beta$ is referred to as the lift of $\tilde{\beta}$. This $\beta$ satisfies $\beta(t+1)=\beta(t)+1$ for all $t \in [0,1]$, and consequently we have, for $t \in [0,1]$, $\beta(t)=\gamma(t)+c $, where $\gamma$ is a warp map of $[0,1]$ and $c \in (0,1]$ (through the identification of $[0,1]$ with $\mathbb{R}/2\pi \mathbb{Z}$). This procedure can be viewed as one that produces a warp map of $\uc$ by `unwrapping' $\uc$ at a chosen point $s$ and generating a warp map of $[0,1]$.
If $\Gamma_I:=\{\gamma:[0,1] \to [0,1]: \gamma'>0,\gamma(0)=0, \gamma(1)=1\}$ is the group of diffeomorphisms of $[0,1]$, the map $\Gamma \mapsto \uc \times \Gamma_I$ is a bijection\footnote{Technically, this is not a bijection since for a $\gamma \in \Gamma$, the corresponding $\beta \in \Gamma_I$ has a jump discontinuity at the point $t_c \in [0,1]$ where $\beta(t_c)+c=1$. This can be circumvented by assuming that the members of $\Gamma$ and $\Gamma_I$ are absolutely continuous (as opposed to diffeomorphims); then the map between to the sets is bijective a.e.}. We will hence employ the product group $\uc \times \Gamma_I$ in place of $\Gamma$. This ensures that the domain of each curve $C_j^m$ and $C^o_i$ can be identified with $[0,1]$ upon unwrapping the circle. 
%This results in $\C^o_i$ becoming periodic with period 1: $\C^o_i(0)=\C^o_i(1)$ for every $i=1,\ldots,n$.

\subsection{Curve completion}
The observed region $\mathcal{R}_j$ associated with a partially observed curve $\C^m_j$ is the subinterval $[0,t_j]$ with $t_j<1$ for all $j=1,\ldots,N$. Suppose that $\C^m_j(0)=\mathbf{a}_j:=(a_{1j},a_{2j})^T$ and $\C^m_j(t_j)=\mathbf{b}_j:=(b_{1j},b_{2j})^T$. Then the set of curves comprising the missing segment of curve $\C^m_j$ is 
$$\mathcal{X}_j:=\{\X: [t_j,1] \to \Rtwo: \X(t_j)=\mathbf{b}, \X(1)=\mathbf{a}\} .$$
For a partially observed $\C_j^m:[0,t_j]$ and an $X \in \mathcal{X}_j$ with $j=1,\ldots,n$, define its completion to be the concatenated closed curve
\begin{equation*}
C_j \circ X(t):= \C_j^m(t) \mathbb{I}_{t \in [0,t_j]}+\X(t) \mathbb{I}_{(t_j,1]}.
\end{equation*}

Denote by $Q_{\C^m_j \circ X}$ the square-root transform of $\C^m_j \circ X$. For each $j=1,\ldots,n$, let $\Theta_j:=\uc \times \Gamma_I \times SO(2) \times \mathcal{X}_j$, and recall that $\mathcal{C}^o$ and $\mathcal{C}^m$ denote the sets of fully and partially observed curves, respectively. For a fixed $C \in \mathcal{C}$, for each $j=1,\ldots,n$ define the cost functional $\Phi_{\theta_j}:\mathcal{C}^m \times \mathcal{C} \to \mathbb{R}$ by
\begin{align*}
\Phi_{\theta_j}(C_j^m,C)&:=d^2(C_j^m \circ X_j, OC(\gamma)), \quad \theta_j \in \Theta_j\\
&=\inf_{(s,\gamma,O) \in \uc \times \Gamma_I\times SO(2)}\|Q_{C_j^m \circ X_j}, OQ_{C}(\gamma)\sqrt{\gamma'}\|_{\ltwo}^2.
\end{align*}
The optimal shape completion of a partially observed curve $Q_{C_j^m}, j=1,\ldots,n$ is $Q_{C_j^m\circ X_j^*}$, where $X_j^*$ is obtained from the solution set of:
\begin{equation}
\label{opt}
\theta^*_j:=(s_j^*,\gamma_j^*,O_j^*,X_j^*)=\argmin_{\theta_j \in \Theta_j} \Phi_{\theta_j}(C_j^m,C).
\end{equation}
In the expression above $s_j^*$ corresponds to the optimal point at which $\uc$ was unwrapped in order to identify $\Gamma$ with $\uc \times \Gamma_I$; $\gamma_j^*:[0,1] \to [0,1]$ and $O_j^*$ represents the optimal reparameterization of the curve $C^m_j \circ X^*_j$. The use of a valid distance on the quotient shape space $\mathcal{Q}_s$ allows us to apply the shape transformations on $Q_{C_i}, i=1\ldots,n$ in the variational problem with introducing any arbitrariness. The (product) group structure of $\uc \times \Gamma_I\times SO(2)$, and its action on $\mathcal{Q}$, ensures that if the transformations were to have been applied to $Q_{C^m_j \circ X^*}$, the resulting solution set would instead contain the corresponding group inverses. 
(INSERT ILLUSTRATIVE FIGURE). 

\subsection{Classification}
We outline two approaches for classification of partially observed curves $C^m_j, j=1,\ldots,N$. 
\subsubsection{Combining completion and classification}
The variational formulation for the completion of each curve $C^m_j$ with respect to a fixed curve $C$ can be augmented to address the classification task in the following manner. Using the training sample $\{(y_i,\C^o_i),1,\ldots N\}$ partition the set $\{C^o_i, i=1,\ldots,N\}$ into $\bigcup_{g \in G}\{C^o_{ig}, i=1\ldots,N_g\}$ with $N_1+\cdots+N_g=N$. Recall that the shape space of a given set of curves is the quotient metric space with the metric $d$ in (\ref{distance}). Such a structure allows us to define the sample Fr\'{e}chet mean shape curve for a given set of curves. Formally, for each $g \in G$, consider the sample Fr\'{e}chet functional define on the set $\mathcal{C}$ as
$$F_g:C \to \mathbb{R}, \quad  C \ni C  \mapsto F_g(C):= \sum_{i=1}^{N_g} d^2(C,C_i^0).$$
The (local) minimizer of $F_g$ is referred to as the Fr\'{e}chet mean set, since $d$ is a distance between two equivalence classes. We then select a member $\hat{M}_g:\uc \to \Rtwo$ from set this and refer to it as the Fr\'{e}chet mean of the group $\{C^o_{ig}, i=1\ldots,N_g\}$ with $g =1,\ldots,G$. 

The variational classifier based on the augmented optimization problem in (\ref{opt}) is defined by the following rule:
\[
\text{Assign $C^m_j$ to group $g^*$ where }  (g^*,\theta^*_j)= \argmin_{g \in \{1,\ldots,G\} \times \theta_j \in \Theta_j} \Phi_{\theta_j}(C^m_j,\hat{M}_g).
\]
The classifier assigns $C^m_j$ to the group whose Fr\'{e}chet mean is closest to $C^m_j$. The advantage of this approach lies in the fact that completion and classification of partially observed curves are unified under the same metric (distance), and in a certain sense carried out simultaneously.  

\subsubsection{Kernel-based classifier}
Consider curves $C_j^m\circ X_j: \uc \to \Rtwo, j=1\ldots,n$ that have been completed using the variational formulation in (\ref{opt}). For each curve $C^m_j \circ X_j$ a nonparametric estimate of the (conditional) group probabilities $\pi_g:=P(y_j=g|C_j^m \circ X_j),g=1,\ldots,G$ can be constructed using the distance $d$ on the shape space of curves. Using the estimate a curve $C_j^m\circ X_j$ is assigned to the group with the largest probability. We eschew the cumbersome notation involving the completed curves and outline the methodology for a generic curve $C:\uc \to \Rtwo$ to belong to two groups based on the training sample; hence $g=1,2$ with labels 0 and 1. Extension to more than two groups is routine. 

Assume that training sample $\{(y_i,C^0_i,i=1\ldots,N\}$ consists of independent realizations of the random element $(\mathbf{y},\mathbf{C})$ taking values in $\{0,1\}\times \mathcal{C}$. For a probability space $(\Omega,\mathcal{F},\mathbb{P})$, the random element $\mathbf{C}$ is the mapping $(t,\omega)\mapsto \mathbf{C}(t,\omega)$. Thus $\mathbf{C}$ is an $\Rtwo$-valued stochastic process $\{\mathbf{C}(t), t \in \uc\}$. The kernel-based estimate (see for e.g. Chapter 8 of \cite{FV}) of the probability of assignment for a curve $C$ to group 1, $\pi(C)$, is given by
\begin{equation}
\label{prob}
\hat{\pi}_N(C)=\frac{\sum_{i=1}^Ny_iK_{i,h_N}}{\sum_{i=1}^N K_{i,h_N}},
\end{equation} 
where $K_{i,h_N}:=K(h_N^{-1}d(C,C_i^o))$ for a given kernel $K$ and a positive bandwidth sequence $h_N$. The distance $d$ is defined on the quotient space of shape curves in (\ref{distance}).

The asymptotic properties of the estimator $\hat{\pi}_N(C)$ are intimately related to the (shifted) small-ball probability of the process $\mathbf{C}$ under the metric $d$: 
\begin{align*}
\phi(Q_C,h_N)&:=\mathbb{P}(d(\mathbf{C},C) < h_N), \quad C \in \mathcal{C}, h_N>0 \\
&=\mathbb{P}\left(\inf_{(O,\gamma) \in SO(2) \times \Gamma}\|\mathbf{Q_C}-OQ_{C}(\gamma)\sqrt{|\gamma'|}\|_{\ltwo}<h_N\right),
\end{align*}
where $\mathbf{Q_C}$ is the (pathwise) square-root transform of the random curve $\mathbf{C}$. For a detailed account of small-ball and shifted small-ball probabilities of processes, and their role in kernel-based estimators involving functional data, see \cite{FV, CR, AM, WL} and references therein. To the best of our knowledge results regarding small-ball probabilities are available only for processes with values in linear function spaces (e.g. Hilbert space with $\mathbb{L}^2$ norm or Banach space with supremum norm). The process $\mathbf{Q_C}=\frac{\mathbf{C'}}{\|\mathbf{C}'\|_{\ltwo}}$ takes values in an infinite-dimensional manifold (pre-shape space) $\mathcal{Q} \subset \mathbb{L}^2(\uc,\Rtwo)$, defined earlier. Moreover, $d$ is on the quotient shape space, and is defined between orbits of $\mathbf{Q_C}$ and $Q_C$ (with respect to the action of $\Gamma$ and $\uc$). 

We can view the class probability $\pi$ (conditional expectation of $\mathbf{y}$ given $\mathbf{C}$) as a map from $\mathcal{C}$ to $[0,1]$. 
We make the following assumptions. 
\begin{enumerate}[\text{A}1.]
\item The kernel $K$ is supported on $[0,1]$ and bounded away from 0 and 1.
\item The bandwidth $h_N \to 0$ as $N \to \infty$.
\item $\phi(C,h_N)>0$ for every $h_N >0$ with $N\phi(C,h_N)\to \infty$ as $N\to \infty$.
\item The conditional probability $\pi:\mathcal{C} \to [0,1]$ is $\alpha$-Lipschitz, i.e. there exists a $\lambda>0$ such that for every $\tilde{C} \in \mathcal{C}$, $|\pi(C)-\pi(\tilde{C})|\leq \lambda \|C-\tilde{C}\|^\alpha$.
\end{enumerate}
The following result relates the behaviour of $\phi(Q_C,h_N)$ to the small-ball probability $\phi(C,h_N)=\mathbb{P}(\|\mathbf{C}-C\|<h_N)$ of the process $\mathbf{C}$ (taking values in a linear space), and establishes consistency and rate of convergence of the estimate $\hat{\pi}_N$, as $N\to \infty$. 
\begin{theorem}
\label{th1}
 Under assumptions A1-A3, $\phi(Q_C,h_N)>0$ for every $h_N>0$ and $N\phi(Q_C,h_N)\to \infty$ as $N\to \infty$. As a consequence, $N \to \infty$:
 \begin{enumerate}[1.]
\item  $\hat{\pi}_N$ converges in probability to $\pi$;
\item $\hat{\pi}_N-\pi=O_\mathbb{P}(h^\beta_N)$.
 \end{enumerate}
\end{theorem}
\begin{proof}
The key argument is to demonstrate that $\phi(Q_C,h_N)>0$ and $N\phi(Q_C,h_N)\to \infty$ under assumptions A1-A3. Proofs of consistency and rate of convergence follow using almost identical arguments as in the proofs of Theorems 6.1 (p. 63) and 8.2 (p. 123) of \cite{FV}, and are omitted. 

%Observe that for $x:\uc \to \Rtwo$, the map $x \mapsto Ox(\gamma)\sqrt{|\gamma'|}$ for $O \in SO(2)$ and $\gamma \in \Gamma$ preserves the squared norm $\|Ox(\gamma)\sqrt{|\gamma'|}\|^2_{\ltwo}$. Indeed
%\begin{align*}
%\int_\uc \|Ox(\gamma(t))\sqrt{|\gamma'(t)|}\|_2^2 dt&=\int_\uc\left[(x(\gamma(t))\sqrt{|\gamma'(t)|})^TO^TO(x(\gamma(t))\sqrt{|\gamma'(t)|})\right]dt\\
%&=\int_\uc \|x(s)\|_2^2ds,
%\end{align*}
%since $O^TO=I_2$ and by a change of variable argument. However, 
%\begin{align*}
%\int_\uc \|Ox(\gamma(t))\sqrt{|\gamma'(t)|}\|_2 dt&=\int_\uc\left[(x(\gamma(t))\sqrt{|\gamma'(t)|})^TO^TO(x(\gamma(t))\sqrt{|\gamma'(t)|})\right]^{1/2}dt\\
%&=\int_\uc \|x(s)\|_2\sqrt{|\gamma'(s)|}ds\\
%&\geq \int_\uc \|x(s)\|_2ds,
%\end{align*}
%since $\sqrt{|\gamma'|}$ is positive, and equality is attained if and only if $\gamma(s)=s$, the identity map. Therefore,
%\begin{align*}
%\|\mathbf{Q_C}-OQ_C(\gamma)\sqrt{\|\gamma'\|}\|_{\ltwo}^2&=\int_\uc \left[\|\mathbf{Q_C}\|_2^2+\|OQ_C(\gamma)\sqrt{|\gamma'|}\|_2^2
%-2\|\mathbf{Q_C}\|_2\|OQ_C(\gamma)\sqrt{|\gamma'|}\|_2\right]dt\\
%&=\int_\uc\|\mathbf{Q_C}(t)\|_2^2dt+\int_\uc \|Q_C(t)\|^2_2dt\\
%&\quad-2\int_\uc \|\mathbf{Q_C}(t)\|_2\|OQ_C(\gamma(t))\sqrt{|\gamma'(t)|}\|_2dt\\
%&\leq \int_\uc \left[\|\mathbf{Q_C}\|_2^2+\|Q_C\|_2^2
%-2\|\mathbf{Q_C}\|_2\|Q_C\|_2\right]dt\\
%&=\|\mathbf{Q_C}-Q_C\|_{\ltwo}^2.
%\end{align*}
The shifted small-ball probability satisfies
\begin{align*}
\phi(Q_C,h_N)&=\mathbb{P}(d(\mathbf{C},C) < h_N), \quad C \in \mathcal{C}, h_N>0\\
&=\mathbb{P}\left(\inf_{(O,\gamma) \in SO(2) \times \Gamma}\|\mathbf{Q_C}-OQ_{C}(\gamma)\sqrt{|\gamma'|}\|_{\ltwo}<h_N\right)\\
&=\mathbb{P}\left(\inf_{(O,\gamma) \in SO(2) \times \Gamma}\|\mathbf{Q_C}-OQ_{C}(\gamma)\sqrt{|\gamma'|}\|_{\ltwo}^2<h^2_N\right)\\
&=\mathbb{P}\left( \|\mathbf{Q_C}-\tilde{O}Q_{C}(\tilde{\gamma})\sqrt{|\tilde{\gamma}'|}\|_{\ltwo}^2<h^2_N \text{ for some } (\tilde{O},\tilde{\gamma}) \in SO(2) \times \Gamma \right)\\
\end{align*}
under the assumption that the (unique) infimum is attained at $(\tilde{O},\tilde{\gamma}) \in SO(2) \times \Gamma$.
The infimum will be attained if the orbits of the elements of $\mathcal{Q}$ are closed under the action of the product group $SO(2) \times \Gamma$. While the orbit under $SO(2)$ is closed, the same isn't generally true for $\Gamma$. A technical adjustment in the definition of $\Gamma$ rectifies this; for details see \cite{LRK}.\footnote{The group $\Gamma$ needs to extended to the semi-group $\tilde{\Gamma}$ that allows for derivatives to be 0 at some points. We could then alter the action of $\tilde{\Gamma}$ on $\mathcal{Q}$ to be just the composition $Q_C(\gamma), \gamma \in \tilde{\Gamma}$, and the arguments in the proof remain valid.}
Observe that $\{\omega\in \Omega: \|\mathbf{Q_C}(\omega)-Q_C(\omega)\|^2_{\ltwo}<h_N^2\}\subseteq\{\omega \in \Omega: \|\mathbf{Q_C}(\omega)-\tilde{O}Q_{C}(\omega)(\tilde{\gamma})\sqrt{\tilde{\gamma}'}\|^2_{\ltwo}<h^2_N \text{ for some } (\tilde{O},\tilde{\gamma}) \in SO(2) \times \Gamma \}$. This can be seen by noting that the relationship is trivially true if $\tilde{\gamma}$ is the identity map in $\Gamma$. Thus we have that
\begin{equation*}
\label{eqs}
\phi(Q_C,h_N)\geq \mathbb{P}\left( \|\mathbf{Q_C}-Q_{C}\|_{\ltwo}^2<h^2_N\right).
\end{equation*}

Consider the square-root map $\mathfrak{C}: \mathcal{C} \to \mathcal{Q}$, $\mathfrak{C}(C)=Q_C$. The map is a bijection between the two spaces \citep{AK}. It is however a complicated map between two Hilbert spaces. Instead of directly dealing with the map in order to relate $\phi(Q_C,\cdot)$ to $\phi(C,\cdot)$, we adopt the following strategy. 

Denote by $\uc^\infty$ the unit sphere in $\mathbb{L}^2(\uc,\Rtwo)$. Note that the pre-shape space $\mathcal{Q}=\Big\{Q_C: \int_\uc \|Q_C\|_2^2=1, \int_\uc Q_C(t)\|Q_C(t)\|_2dt=0\Big\}$ is proper subset of $\uc^\infty$. Thus we can view $\mathfrak{\mathbf{C}}(C)=\mathbf{Q_C}$ and $\mathfrak{C}(C)=Q_C$ as random elements taking values in $\uc^\infty$.
Consider the radial map in a Hilbert space $\mathfrak{R}:\mathcal{C}\to \uc^\infty$ given by
\begin{equation*}
\mathfrak{R}C = \left\{ \begin{array}{ll}
         C & \mbox{if $\|C\| \leq  1$};\\
        \frac{C}{\|C\|} & \mbox{if $\|C\|>1$}.\end{array} \right. 
\end{equation*}
The map $\mathfrak{R}$ is the unique metric projection of $\mathcal{C}$ onto $\uc^\infty$ and is 1-Lipschitz and nonexpansive, that is, 
\[
\|\mathfrak{R}C_1 -\mathfrak{R}C_2\| \leq \|C_1-C_2\|, \quad C_1,C_2 \in \mathcal{C}.
\]
Since the image of $\mathfrak{C}$ (i.e. $\mathcal{Q}$) is contained in the image of $\mathfrak{R}$ (i.e. $\uc^\infty$), and the noting that $\mathfrak{C}$ is bijective and $\mathfrak{R}$ is the unique projection on $\uc^\infty$, for $C \in \mathcal{C}$ we necessarily have $\mathfrak{C}(C)=\mathfrak{R}(C)$.

From the definitions of the maps $\mathfrak{C}$ and $\mathfrak{R}$, from equation (\ref{eqs}) we have,
\begin{align*}
\phi(Q_C,h_N)&\geq\mathbb{P}\left(\|\mathfrak{C}(\mathbf{C})-\mathfrak{C}(C)\|_{\ltwo}^2<h^2_N\right)\\
&=\mathbb{P}\left(\|\mathfrak{R}(\mathbf{C})-\mathfrak{C}(C)\|_{\ltwo}^2<h^2_N\right)\\
&\geq \mathbb{P}\left(\|\mathbf{C}-C\|_{\ltwo}^2<h^2_N\right)\\
&=\phi(C,h_N),
\end{align*}
since for a fixed $N$, $\{\omega\in \Omega: \|\mathbf{C}(\omega)-C\|_{\ltwo}<h_N\}\subseteq\{\omega \in \Omega: \|\mathfrak{R}(\mathbf{C}(\omega))-\mathfrak{R}(C)\|_{\ltwo} <h_N|\}$ with $\mathfrak{R}$ being 1-Lipschitz. This completes the proof. 
\end{proof}
\bibliography{biblio}
\bibliographystyle{plainnat}
\end{document}